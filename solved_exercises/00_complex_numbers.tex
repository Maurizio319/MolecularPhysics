\section{Complex numbers}

  \subsection{Conversion in different forms}
  Rewrite $\cos(3-2i)$ in standard form.
  \begin{align*}
    \cos(3-2i) & = \frac{e^{i(3-2i)} + e^{-i(3-2i)}}{2}
               & \text{since }\cos\phi = \frac{e^{i\phi} + e^{-i\phi}}{2} \\
               & = \frac{e^{3i}e^{2} + e^{-3i}e^{-2}}{2}
               & \text{distributive property} \\
               & = \frac{(\cos3 + i\sin3)e^{2} + (\cos3 - i\sin3)e^{-2}}{2}
               & \text{since } e^{i\phi} = \cos\phi + i\sin\phi \\
               & = \cos3\left(\frac{e^2+e^{-2}}{2}\right) + i\sin3\left(\frac{e^2-e^{-2}}{2}\right)
               & \text{collecting } \cos3 \text{ and } i\sin3 \\
               & = \underbrace{\cos3\cosh2}_{\text{real part}} + \underbrace{i\sin3\sinh2}_{\text{immaginary part}}
               & \text{since } \cosh x = \frac{e^x+e^{-x}}{2}, \sinh x = \frac{e^x-e^{-x}}{2} \\
  \end{align*}

  Rewrite $z = 3 - 4i$in polar form.
  \begin{align*}
    r       & = \sqrt{3^2 + (-4)^2} = 5
            & \text{modulus} \\
    \phi    & = \arctan(-4/3)
            & \text{angle on Argand plane} \\
            & \text{Therefore} \\
    z       & = 5e^{\arctan(-4/3)}\\
  \end{align*}

  Show that $|z| = \sqrt{zz^*}$, where $z$ is any complex number.
  \begin{align*}
    \sqrt{zz^*} & = \sqrt{re^{i\phi}re^{-i\phi}}
                & \text{converting into polar form } z \text{ and } z^* \\
                & = \sqrt{r^2e^{i\phi -i\phi}}
                & \text{collecting } r \text{ and using exponent rules} \\
                & = \sqrt{r^2e^{0}} \\
                & = \sqrt{r^2} \\
                & = r = |z|
  \end{align*}

  Show that $A\cos(wt + \phi) = A_1\cos(wt) + A_2\sin(wt) = B_1e^{iwt} + B_2e^{-iwt}$.
  \begin{align*}
    A\cos(wt+\phi)  & = A\cos(wt)\cos(\phi) - A\sin(wt)\sin(\phi)
                    & \text{because } \cos(A+B) = \cos(A)\cos(B) - \sin(A)\sin(B)\\
                    & = A_1\cos(wt) + A_2\sin(wt)
                    & \text{with } A_1 = A\cos(\phi), A_2=-A\sin(\phi)\\
                    & = A_1\left(\frac{e^{iwt}+e^{-iwt}}{2}\right) + A_2\left(\frac{e^{iwt}-e^{-iwt}}{2i}\right)
                    & \text{since } \cos(z)=\frac{e^{iz}+e^{-iz}}{2}, \sin(z)=\frac{e^{iz}-e^{-iz}}{2i} \\
                    & =\left(\frac{A_1}{2}+\frac{A_2}{2i}\right)e^{iwt}+\left(\frac{A_1}{2}-\frac{A_2}{2i}\right)e^{-iwt}
                    & \text{rearranging} \\
                    & = B_1e^{iwt} + B_2e^{-iwt}
                    & \text{with } B_1=\left(\frac{A_1}{2}+\frac{A_2}{2i}\right),B_2=\left(\frac{A_1}{2}-\frac{A_2}{2i}\right)
  \end{align*}

  \subsection{Multivalued functions}
  Compute $1^{\frac{1}{6}}$, i.e. find all roots of the equation $z^{6} = 1$ in $\mathbb{C}$.
  \begin{align*}
    1^{\frac{1}{6}} & = e^{\frac{1}{6}\ln(1)}
                    & \text{since } a^z = e^{z\ln(a)}\\
                    & = e^{\frac{1}{6}2i\pi k}
                    & \text{since } \ln(re^{i\phi}) = \ln(r) + i\phi + 2i\pi k,
                                    r = |\sqrt1| = 1,
                                    \phi = \arctan{0} = 0\\
                    & = e^{\frac{1}{3}i\pi k} \,\,\forall k \in \mathbb{N}^* \wedge k < 6
                    & \text{where } r = 1, \phi' = \frac{1}{3} \pi k
  \end{align*}
  When represented on an Argand plane, these complex numbers are vectors of lenght 1, and the corresponding angles are the ones obtained by substituting the values of k in $\phi'$, namely the set $\{0, \frac{\pi}{3}, \frac{2\pi}{3}, \pi, \frac{4\pi}{3}, \frac{5\pi}{3}\}$.

  Compute $i^i$.
  \begin{align*}
    i^i & = e^{i\ln i}
        & \text{since } a^z = e^{z\ln(a)} \\
        & = e^{i\left(\frac{i\pi}{2} + 2i\pi k\right)}
        & \text{since } \ln(re^{i\phi}) = \ln(r) + i\phi + 2i\pi k,
                        r = |\sqrt1| = 1,
                        \phi = \lim_{x \to \infty}\arctan{x} = \frac{\pi}{2} \\
        & = e^{-\frac{\pi}{2}-2\pi k}
        & \text{collecting } i\pi \text{ and simplifying } i^2 = -1 \\
        & = e^{-\frac{\pi}{2}}
        & \text{since } 2\pi = 0 \text{, thus you can simplify the k rotations} \\
        & \approx 0.2079
  \end{align*}
  Notice that $i^i$ is a unique real value, despite being defined using only $i$.

  Compute $1^i$.
  \begin{align*}
    1^i & = e^{i\ln 1}
        & \text{since } a^z = e^{z\ln(a)} \\
        & = e^{i0}
        & \text{since } \ln1 = 0 \\
        & = e^{0} = 1
  \end{align*}
  Notice the fact that "$1$ elevated to any power is equal to $1$" holds even in $\mathbb{C}$.
