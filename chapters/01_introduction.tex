\graphicspath{{chapters/01/}}

\chapter{Revisiting classical mechanics}

\section{Physical theories}

  \subsection{Experiment}
  An experiment performed on a physical system is a way to measure observable quantities at a determined time: $O_1(t), \dots, O_n(t)$.
  By measuring more and more observables at the same time, the instantaneous state of the system becomes increasingly characterized.
  A maximum set of observables that leads to the complete characterization of the instantaneous state can be assumed.

    \subsubsection{Example - particle of mass $m$ subject to an harmonic force}
    A particle of mass $m$ subject to an harmonic force like one of a spring is completely characterized by:

    $$(\vec{r}(t),\vec{v}(t))=\vec{r}(t)\in \mathbb{R}^6$$

  \subsection{Definition}
  A physical theory is a mathematical scheme to predict the state of the system, the outcome of future observations: $O_1(t'), \dots, O_n(t')$.
  In particular, an equation that can be used to compute the state at $t'$ given the state at $t$ is called an equation of motion.
  A physical theory is composed by the following elements:

  \begin{multicols}{2}
  \begin{itemize}
      \item definition of the state of a system
      \item recipe to determine the system's equation of motion
      \item prediction of an observable quantity given the state of the system
    \end{itemize}
  \end{multicols}


\section{Classical mechanics}
In classical mechanics, the state of a system is a collection at time $t$ of all the positions of all the particles in the system.
We also need to define the velocity of the particle, or, better, the momentum.
The latter is obtained, in classical mechanics, by scaling the velocity by a constant being in this way equivalent to it.
The mathematical tool that allows to understand how a state evolves in time is the time derivative of the equation of interest.
Solving a classical mechanics problem means to solve a set of differential equations.

  \subsection{Newtonian mechanics}
  Newtonian Mechanics is a theory described by the three Newton's laws, which are in principle well stated:

  \begin{multicols}{2}
    \begin{enumerate}
      \item A body will continue its state at rest or at constant speed unless a force acts.
        This is only true in inertial frames.
        The lack of an initial consensus inertial frame raises issues.
      \item How does the body change when it is subject to a force.
        It's difficult to define exactly what is a force: here we only deal with mass and acceleration.
        We have an issue, stuck in a loop.
      \item Whenever object $A$ acts on object $B$, $B$ acts on $A$ with opposite sign.
        This is the action-reaction principle.
    \end{enumerate}
  \end{multicols}


  \subsection{Example - point particle moving in $1D$ and subject to an harmonic force}
  For a point particle moving in $1D$ subject to an harmonic force (bound to a spring for example) follows the Newton's law for its equation of motion:

  $$\frac{d{^2}}{d{t^2}}x(t) = -\frac{k}{m}x(t)$$

  If, given $\lambda$ the amplitude, $L$ the oscillation and $L_0$ the distance of the point from the origin of the axis at time $0$:

  \begin{multicols}{3}
    \begin{itemize}
      \item $\frac{\lambda}{L}\ll 1$
      \item $\frac{\lambda}{L_0}\ll 1$.
      \item $\frac{L}{L_0}\ll 1$.
    \end{itemize}
  \end{multicols}

  This is a second-order differential equation such that $\frac{d{^2}}{d{t^2}}f(t) = -\frac{k}{m}f(t)$.
  There are only two solutions:

  \begin{align*}
    f_1(t) = \sin(\omega t) &\Rightarrow \frac{d{^2}}{d{t^2}}f_1(t) = -\omega^2f_1(t)\\
    f_2(t) = \cos(\omega t) &\Rightarrow \frac{d{^2}}{d{t^2}}f_2(t) = -\omega^2f_2(t)\\
  \end{align*}

  Clearly, $f_1(t)$ and $f_2(t)$ are solutions if $\omega^2 = \frac{k}{m}$.
  Neither of them is a good solution: the first one is only valid if the particle starts at the origin, while the second one only holds if the particle is at rest at time $0$.
  So the most general solution is:

  $$x(t) = A\cos\sqrt{\frac{k}{m}}t+B\sin\sqrt{\frac{k}{m}}t$$

  To find $A$ and $B$, information about the initial conditions are used.
  Let the initial position $x(0) = x_0$.
  Then

  $$x(0) = A = x_0$$

  Considering initial velocity: $\frac{d{}}{d{t}}x(t)|_{t=0} = v_0$, then:

  \begin{align*}
    \frac{d{}}{d{t}}x(t) &= -\sqrt{\frac{k}{m}}A\sin\sqrt{\frac{k}{m}}t + \sqrt{\frac{k}{m}}B\cos\sqrt{\frac{k}{m}}t\\
    v_0 &= \sqrt{\frac{k}{m}} B \Rightarrow B = \sqrt{\frac{m}{k}}v_0
  \end{align*}

  So the final solution is:

  $$\begin{cases}
    x(t) = x_0\cos\sqrt{\frac{k}{m}}t+\sqrt{\frac{m}{k}}v_0\sin\sqrt{\frac{k}{m}}t\\
    v(t) = v_0\cos\sqrt{\frac{k}{m}} t -x_0\sqrt{\frac{k}{m}}\sin\sqrt{\frac{k}{m}} t
  \end{cases}$$

  \subsection{Phase-space}
  It is convenient to plot the solutions on the phase space, a plane such that on the $x$-axis there is the position $x=q(t)$ and on the $y$-axis the momentum $m\times v=p(t)$.
  In this way the state of the system will be described as:

  $$\Gamma(t) = (q(t),p(t))$$

  Cauchy's theorem implies that given a n-th order differential equation and given $n$ initial conditions, there exists exactly one solution.
  Its corollary implies that trajectories in the phase-space can never intersect.
  Because of this classical mechanics is fully deterministic, as future $x(t)$ and $v(t)$ can be unambiguously predicted.

  \subsection{Systems in three dimension and with more than one particle}
  For systems with $D=3$ and for more than one particle the equation of motion is:

  $$\begin{cases}
    m_1 \frac{d{^2}}{d{t^2}}\vec{r}_1(t) = \vec{F}_1(\vec{r}_1(t), \dots,\vec{r}_N(t))\\
    \vdots\\
    m_N \frac{d{^2}}{d{t^2}}\vec{r}_N(t) = \vec{F}_N(\vec{r}_1(t), \dots,\vec{r}_N(t))\\
  \end{cases}$$

  Correspondingly, the phase-space is $6N$ dimensional.
  Moreover, for the $N$ vector equations there are $N$ scalar ones.

  \subsection{Work and energy}
  Let $\vec{r}$ be a trajectory followed by a particle subject to a force $\vec{F}$.
  The work of the force $\vec{F}$ from point $A$ to point $B$ of the trajectory is defined as:

  $$W_{AB} = \int_{\vec{r}_a}^{\vec{r}_b}d \vec{r}\cdot \vec{F}$$

  The kinetic energy of the particle is instead:

  $$T = \frac{1}{2}mv^2$$

  Work and energy are related:

  \begin{align*}
    \frac{d{}}{d{t}}T = \frac{d{}}{d{t}}\frac{1}{2}mv^2 = \frac{1}{2}m \frac{d{}}{d{t}}v^2 = \frac{1}{2}2m \vec{v}\underbrace{\frac{d{\vec{v}}}{d{t}}}_{\vec{a}} = \vec{v} \vec{F}\\
    \int_{t_0}^{t_f}dt \frac{d{}}{d{t}}T = T_B-T_A = \int_{t_i}^{t_f}dt \frac{d{\vec{r}}}{d{t}}\vec{F} = \int_{\vec{r}_A}^{\vec{r}_B} d \vec{r}\cdot \vec{F} = W_{AB}\\
    T_B-T_A = W_{A\rightarrow B}
  \end{align*}

    \subsubsection{Conservative forces}
    For conservative forces, the work from point $A$ to point $B$ does not depend on the path followed. In particular, for each path $1$ and $2$: $W_{AB}^1 = W_{AB}^2$ and:

    $$-W_{AB} = U(\vec{r}_B) - U(\vec{r}_A)$$

    Where $U(\vec{r})$ is the potential energy.
    In one dimension:

    \begin{align*}
      U(r) -U(r_0) = -w_{x_0x} = -\int_{x_0}^xdyF(y) \Rightarrow\\
      -\frac{d{}}{d{x}}U(x) = F(x)
    \end{align*}

      \paragraph{Three dimensional case}

      $$\begin{cases}
        F_x(x,y,z) = - \frac{\partial {}}{\partial {x}}U(x,y,z)\\
        F_y(x,y,z) = - \frac{\partial {}}{\partial {y}}U(x,y,z)\\
        F_z(x,y,z) = - \frac{\partial {}}{\partial {z}}U(x,y,z)\\
      \end{cases}$$

      Or, in short hand notation, let $\vec{r}=(x,y,z)$ and $\vec{\nabla}(\frac{\partial {}}{\partial {x}},\frac{\partial {}}{\partial {y}},\frac{\partial {}}{\partial {z}})$, then:

      $$\vec{F}(\vec{r}) = -\vec{\nabla}U(\vec{r})$$

      \paragraph{Central forces}
      Central forces are a notable class of conservative forces, for which $\vec{F}(\vec{r}) = \hat{\omega}_{r}f(r)$ and $\vec{r} = \hat{U}_{r}|\vec{r}| = \hat{U}_{r}r$.
      Some examples:

      \begin{multicols}{2}
        \begin{itemize}
          \item Coulomb: $\vec{F}_e = \hat{U}_{\vec{r}}\frac{q_1q_2}{r_{12}^2}$
          \item Gravity: $\vec{F}_G = -\hat{U}_r\frac{M_1M_2}{r_{12}^2}G$
          \item Harmonic $\vec{F} = -\hat{\omega}_r(\vec{r}-\vec{r}_0)$
          \item $\cdots$
        \end{itemize}
      \end{multicols}

  \subsection{Conservation of mechanical energy}
  Reconsidering the relationships between $T$ and $W$:

  $$T_B-T_A = W_{A\rightarrow B} = U_A-U_B$$

  The mechanical energy $H$ can be introduced, such that:

  $$H_A = T_A+U_A = T_B+U_B = H_B$$

  The mechanical energy is conserved in the system only if conservative forces act on it.
  Energy conservation allows us to solve Newton's equation, which is generally impossible to handle.
  This is because conservation laws help gaining partial information without having to solve Newton's equation.
  At this level, energy conservation comes in as a matter of convenience.

    \subsubsection{Example}
    Consider a cart going down a path with a loop.
    Let $A$ be the starting highest point when it starts going and $B$ the lowest point where it stops accelerating.

    $$\begin{cases}H_A = \underbrace{T_A}_{=0}+\underbrace{U_A}_{=mgh}\\
    H_B = \underbrace{T_B}_{-\frac{1}{2}mv^2}+\underbrace{U_B}_{=0}\end{cases}
    \Rightarrow v_B = \sqrt{2gh}$$

  \subsection{Angular momentum conservation}
  Let the angular momentum:

  $$\vec{L}(t) = \vec{r}(t)\times m \vec{v}(t)$$
  The angular momentum is also defined as the vector product of the particle position and its linear momentum.

  $$\vec{L}=\bar{x} \times \bar{P}$$

  There are two ways to solve a cross product: $\vec{a}\times \vec{b}\perp \vec{a}$, $\vec{a}\times \vec{b}\perp \vec{b}$ and $|\vec{a}\times \vec{b}| = |\vec{a}||\vec{b}|\sin\theta$.
  This implies that $\vec{a}\parallel \vec{b}\Rightarrow \vec{a}\times \vec{b} = 0$.
  Now considering the vectors' coordinates:

  $$\vec{a}\times \vec{b} = \hat{i}(a_yb_z - a_zb_y) + \hat{j}(a_xb_z - a_zb_x) + \hat{k}(a_xb_y-a_yb_x)$$

  By taking the derivative of the angular momentum with respect to time, we can asses under which conditions linear momentum is conserved.
  Consider that velocity and momentum are parallel, therefore the cross product depending on the angle between them is zero:

\begin{align*}
    \frac{d{}}{d{t}}\vec{L} &= \frac{d{}}{d{t}}(\vec{r}\times \vec{p})=\\
                            &=\underbrace{\frac{d{\vec{r}}}{d{t}}}_{=0}\times\vec{p} + \underbrace{\frac{d{\vec{p}}}{d{t}}}_{=F}\times\vec{r} =\\
                            &= \vec{r}\times \vec{F}
  \end{align*}

  Whenever the torque of a particle is zero, then the angular momentum of that particle is conserved.
  This is the case for centripetal and centrifugal forces.

  \begin{align*}
    \frac{d \vec{L}}{d t}&=\bar{x} \times \vec{F}=0 \\
   \end{align*}

  The following is true for any force:

   \begin{align*}
    \frac{d}{d t}\left(\vec{L}_{\text {TOT }}\right)&=\frac{d}{d t}\left(\Sigma_{i} \vec{L}_{i}\right)=0 \\
  \end{align*}

  If $\vec{r}\parallel \vec{F}\Rightarrow \frac{d{}}{d{t}}\vec{L} = 0$.
  Therefore, for a conservative force there is angular momentum conservation.
  Summarizing, the motion in a central force conserves energy and angular momentum.

\section{Hamiltonian formulation of mechanics}
In the Newtonian formulation, the fundamental aspect is the force:

$$m \vec{a} = \vec{F}$$

Defining a physical theory in classical mechanics corresponds to specifying what is a force.

  \subsection{Hamilton's theory}
  Hamilton's theory is an equivalent reformulation of mechanics in which the key concept is not the force, but the Hamiltonian $H$, which is closely related to energy.
  While from a practical standpoint the two formulation are equivalent with identical equations of motion, modern physics has shown that the notion of energy is more fundamental than the one of force.
  In this formulation the derivative of the position will be the derivative of $H$ with respect to position.
  It can be noted that from $H(p,q)$ it is possible to retrieve Newton's equations of motion, solve them and obtain predictions to compare with experimental results following:

  \begin{multicols}{2}
    \begin{enumerate}
        \item Define H.
        \item Compute equation of motion.
        \item Solve equation of motion.
        \item Use solution to obtain predictions of experimentally observable quantities.
        \item Experimentally test the prediction.
    \end{enumerate}
  \end{multicols}

  This procedure is repeated until concordance with experiments is reached.
  In this formulation the concept of force is never defined.

  \subsection{Hamilton's equations}
  From this point on, energy will be identified in an Hamiltonian:

  $$H(\underbrace{\vec{p}}_{\text{momentum}}, \underbrace{\vec{q}}_{\text{position}}) = \frac{\vec{p}^2}{2m} + U(\vec{q})$$

  So the equations of motion become:

  $$\begin{cases}
    \dot{q} = + \frac{\partial {H}}{\partial {p}}\Rightarrow \dot{q} = \frac{p}{m}\Rightarrow p = \dot{q}m\\
    \dot{p} = - \frac{\partial {H}}{\partial {q}}\Rightarrow \dot{p} = -\frac{\partial {U(q)}}{\partial {q}}\Rightarrow ma = F
  \end{cases}$$

  The Hamilton's equation describes directly the evolution of a point of phase space and the state of the system.
  Let $\Gamma = (p,q)$ and $H = H(\Gamma)$, then:

  $$\begin{cases}
    \dot{\Gamma}_1 = +\frac{\partial {H}}{\partial {\Gamma_2}}\\
    \dot{\Gamma}_2 = +\frac{\partial {H}}{\partial {\Gamma_1}}
  \end{cases}$$

  % To add definitions, remember to put in the prefix
  % the command \newtheorem{definition}{Definition}
  \begin{definition}
    \textbf{Phase space:}
    In dynamical system theory, a phase space is a space in which all possible
    states of a system are represented, with each possible state corresponding
    to one unique point in the phase space. For mechanical systems, the phase
    space usually consists of all possible values of position and momentum variables.
  \end{definition}

  For many particles in three dimensions:

  $$\Gamma\overbrace{(\underbrace{\vec{p}_1,\dots,\vec{p}_N}_{\vec{P}\in \mathbb{R}^{3N}};\underbrace{\vec{q}_1,\dots,\vec{q}_N}_{\vec{Q}\in \mathbb{R}^{3N}})}^{\Gamma\in \mathbb{R}^{6N}}$$

  The state of a many body system is described by the evolution of a point in a large dimensional vector space.

  \subsection{Harmonic oscillator}
  Consider $\omega = \sqrt{\frac{k}{m}}$:

  $$\begin{cases}
    p(t) = -\omega q_0\sin\omega t + v_0\cos\omega t\\
    q(t) = q_0\cos\omega t + \frac{v_0}{\omega}\sin\omega t
  \end{cases}$$

  Now from that:

  \begin{align*}
    \frac{p^2}{\omega^2}+q^2 &= q_0^2\sin^2\omega t + \frac{v_0^2}{\omega^2}\cos^2\omega t - 2 \frac{q_0v_0}{\omega}\sin \omega t\cos \omega t + q_0^2\cos^2\omega t + \frac{v_0^2}{\omega_2}\sin^2\omega t + 2\frac{q_0v_0}{\omega}\sin \omega t\cos \omega t =\\
                             &=(q_0^2+\frac{v_0^2}{\omega^2}
  \end{align*}

  The problem has a general structure of $\frac{q^2}{A}+\frac{p^2}{B}=1$ and the trajectories draw ellipses in the phase space.

\section{Classical theory of the hydrogen atom}
The classical theory of the hydrogen atom is defined by the classical Bohr model.
The hydrogen atom is formed by a proton in the centre with an electron orbiting around it.

  \subsection{Approximations}
  The classical Bohr model makes some approximation to better model the atom:
  \begin{multicols}{2}

    \begin{itemize}
      \item The mass of the electron divided by the mass of the proton is around $\frac{1}{2000}$ ($\frac{m_e}{M_p}\approx \frac{1}{2000}$).
        The center of mass of the system is very close to the nucleus, therefore we can assume that the nucleus remains still.
        For the sake of simplicity, an infinite proton mass $\frac{m_e}{M_p}=0$ is assumed.
      \item The charge of the proton is the negative of the charge of the electron, so the overall system is neutral.
      \item  Because $\frac{d{}}{d{t}}\vec{L}=0$ (angular momentum conservation), the electron's motion occurs on a plane and is two dimensional.
    \end{itemize}
  \end{multicols}

  \subsection{Newton's equation}
  Let us start from Newton's equations.
  We wish to express the attracting force acting on the electron towards the proton.
  This force has an inward direction with negative sign.
  Since the electron coordinates $x$, $y$, $z$ are varying with time, they can be expressed as $x(t)$, $y(t)$ and $z(t)$.
  The equations are coupled, meaning that we cannot solve one at a time: $y(t)$ and $z(t)$ are required for solving the first one and the same applies to the others.
  This is a quite tricky task, but luckily there is a way round this problem. We can employ the energy conservation that we mentioned previously though the usage of polar coordinates.


  \subsection{Mechanical energy in polar coordinates}

  $$H = \underbrace{\frac{p^2}{2m}}_{\text{kinetic energy}}-\underbrace{\frac{e^2}{r}}_{\text{potential energy}}$$

  In this case we have a position vector $r$ going from the proton to the electron.
  The Coulombic force is a radial centripetal force: it points on the same line of the position, so $F$ is parallel to $r$.
  The mathematical expression shows that forces are anti-parallel.
  Rather than using Cartesian coordinates, we can use rotating coordinates.
  The centrality of force will imply two main results: mechanical energy is conserved and the angular momentum  of the particle is the torque, but it is zero for centripetal force, so $\vec{L}$ is conserved.
  The second considerations lead us to understand that orbit lines are confined on a plane. This is particularly relevant, as in a plane we only need two coordinates, as the third coordinate is fixed to zero.
  The potential energy is derived from Coulomb's $F=-\hat{r}\frac{e^2}{|\vec{r}^2|}$, so $U(|\vec{r}|) = -\frac{e^2}{|\vec{r}|}$
  Now, by using conservation to replace the differential equation, the position is divided in two components $\hat{u}_r$ and $\hat{u}_\theta$ orthogonal to each other, so that $v^2 = v_r^2+v_\theta^2$:

  $$\vec{v} = \hat{u}_\theta v_\theta +\hat{u}_r\underbrace{v_r}_{\frac{d{r}}{d{t}}}\Rightarrow v^2 = \biggl(\frac{d{r}}{d{t}}\biggr)^2+v_\theta^2$$

  Now, rewriting the angular momentum:

  \begin{align*}
    \vec{L} = m \vec{r}\times(\underbrace{\vec{v}_r}_{=0}+\vec{v}_\theta) &= m \vec{r}\times\vec{v}_\theta\\
    |L| &= mrv_\theta\\
    L^2 = m^2r^2v_\theta^2
  \end{align*}

  Substituting this in the Hamiltonian:

  \begin{align*}
    H &= \frac{1}{2}mv_r^2 + \frac{1}{2}mv_\theta^2 - \frac{e^2}{r}=\\
      &=\frac{1}{2}mv_r^2 + \underbrace{\frac{L^2}{2mr^2}}_{\text{Simil-potential, L constant}} - \frac{e^2}{r}
  \end{align*}

  This term depends on $r$ and not on $\theta$ and effectively looks like a potential energy.
  So $\frac{L^2}{2mr^2}-\frac{e^2}{r}\equiv V_{eff}(r)$.
  And:

  $$H = \frac{1}{2}m\biggl(\frac{d{r}}{d{t}}\biggr)^2+V_{eff}(r)$$

  By using polar coordinates and angular momentum conservation, the mechanical energy has been written in the form of an effective one dimensional system with $U(r) \rightarrow V_{eff}(r)$.
  Using this trick, it is immediate to infer the qualitative structure of orbits.

  \subsection{Case 1 - $E>0$}
  The case in which $E>0$ is the case of an unbound orbit.
  The electron approaches the proton and accelerates.
  There is an inversion point and:

  $$\begin{cases} H = \frac{1}{2}m\biggl(\frac{d{r}}{d{t}}\biggr)^2+V_{eff}(r)\\
    \parallel\\
    E<V_{eff}(r)
  \end{cases}
  \Rightarrow \biggl(\frac{d{r}}{d{t}}\biggr)^2< 0$$

  And that is impossible.

  \subsection{Case 2 - $E<0$}
  The case in which $E<0$ is the case of the bound orbit.
  The electron is trying to escape the proton and there are two inversion points.

  \subsection{Conclusion}
  Whenever a charge changes its velocity it emits $\frac{e}{m}$ radiations.
  The energy loss per unit time is:

  $$P = \frac{2}{3}\frac{m_er_e}{c}a^2\qquad a = \omega^2 r$$

  According to Larmer's law, the electron would spiral into the nucleus in $10^{-15}s$.
  Because of this, classical atoms are unstable.
