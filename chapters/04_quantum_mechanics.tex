\chapter{Quantum mechanics}

\section{State of a system}
The instantaneous state of a system is represented by points in a Hilbert space: $|\psi(t)\rangle$.
More precisely,  states are associated to rays - since $a|\psi\rangle$ and $|\psi\rangle$ represent the same state.
Let $|\psi_3\rangle = a |\psi_2\rangle$, $|\psi_1\rangle$ and $|\psi_2\rangle$ represent different states and $|\psi_1\rangle$ and $|\psi_3\rangle$ represent the same one.\\
\noindent
The wavefunction $\psi(Q,t)$ describes the state of a quantum system, where $Q=\vec{q_1},...,\vec{q_N}$ are coordinates for the position of all the particles.
Keep in mind that $\psi$ cannot be measured, it is a mere mathematical tool that we use for computing and making predictions.

\section{Observable quantities}
Observable quantities are associated to an Hermitian Operator, for example:

\begin{multicols}{2}
	\begin{itemize}
		\item $\underbrace{x}_{\text{position}}\rightarrow\underbrace{\hat{x}}_{\text{position operator}}$
		\item $\underbrace{h}_{\text{energy}}\rightarrow\underbrace{\hat{H}}_{\text{Hamiltonian operator}}$
	\end{itemize}
\end{multicols}

The only exception is time, which is not associated to an operator.

$$O(p,q)\rightarrow\hat{O}(\underbrace{-i\hbar\vec{\nabla}}_{\text{p}},\bar{q})$$

\section{Outcomes of measurement}
The possible outcomes of measurement of an observable $O$ are the eigenvalues of the corresponding operator $\hat{O}$:

$$\hat{O}|o_n\rangle = o_n|o_n\rangle$$

If the system is in the quantum state $|\psi\rangle$, the probability of finding the value $o_n$ when measuring $\hat{O}$ is:

$$P(o_n) = |\langle o_n|\psi\rangle|^2$$

Immediately after the measurement of $\hat{O}$ , yielding $o_n$, the state of the system is described by:

$$\psi(Q,t_0)=f_n^{\circ}(Q)$$

By measuring we observe the \textbf{collapse of the wavefunction} e.g. the state of the cat collapses to either dead or alive when we open the box.

\section{Expectation value}
The average over many measurement of $\hat{O}$ or expectation value is given by:

$$\langle \hat{O}\rangle = \frac{\langle\psi|\hat{O}|\psi\rangle}{\langle \psi|\psi\rangle}=\frac{\int d^{3 N} Q \psi^{*}(Q, t) \hat{O} \psi(Q, t)}{\int d_{Q}^{3 N} \psi^{*}(Q, t) \psi(Q, t)}$$

As a corollary,  the mean-square deviation of the measurement is:

$$\Delta^2\hat{O} = \frac{\langle \psi|\hat{O}^2|\psi\rangle}{\langle\psi|\psi\rangle}-\biggl(\frac{\langle\psi|\hat{O}|\psi\rangle}{\langle\psi|\psi\rangle}\biggr)^2$$

It $|\psi\rangle$ is an eigenstate of $\hat{O}$ $|\psi\rangle = |o_n\rangle$, then $\Delta^2\hat{O} = 0$

\section{Time evolution}
The time evolution of $|\psi(t)\rangle$ is described by the Schr\"odinger equation:

$$i\hbar \frac{d{}}{d{t}}|\psi(t)\rangle = H|\psi(t)\rangle$$

The formal solution of this equation is given by the time evolution operator:

$$|\psi(t)\rangle = \hat{U}(t-t_0)|\psi(t_0)\rangle$$

$$\hat{U}(t-t_0) = e^{-\frac{i}{\hbar}\hat{H}(t-t_0)}$$

\section{Remarks}
$\hat{O}_1,\hat{O}_2,...,\hat{O}_N$ are Hermitian operators.\\
Let ${O_l}$ where ${l=1,...,M<N}$ be the subset of mutually commuting operators. Then, for $[O_l,O_m]=0$ with $O_l,O_m \in M$ there will be a common set of eigenstates.\\
\noindent
$\hat{O}_lf_i=o_i^{(l)}f_i$, where $f_i$ are the eigenstates.
By measuring immediately after a previous measurement $M,$ the result will be equal, as the $M$ measurement will provide the maximum information possible on the quantum system.
Any other measurement will destroy the information e.g. if we measure the momentum (non-commuting) after the position, we will not know the position any more.

	\subsection{Proof for non commutation of momentum's components}
	Check whether $L_x$ and $L_y$ do not commute.

	$$\bar{L}=\begin{vmatrix}\hat{i}& \hat{j}& \hat{k} \\ x& y& z \\ p_{x}& p_{y}& p_{z}\end{vmatrix}=\hat{i} \underbrace{\left(y p_{z}-z p_{y}\right)}_{L_{x}}+\hat{j}\underbrace{(z p_{x}-x p_{z})}_{L_{y}}+\hat{k} \underbrace{\left(x p_{y}-y p_{x}\right)}_{L_{z}}$$

	Quantization leads to:

	\begin{align*}
		\hat{L}_{x}&=-i \hbar\left(y \frac{\partial}{d z}-z \frac{\partial}{\partial y}\right) \\
		\hat{L}_{y}&=-i \hbar\left(z \frac{\partial}{\partial x}-x \frac{\partial}{\partial z}\right)\\
		\hat{L}_{z}&=-i \hbar\left(x \frac{\partial}{\partial y}-y \frac{\partial}{\partial x}\right)
	\end{align*}

	Find $\left[\hat{L}_{x}, \hat{L}_{y}\right]$:

	$$\underbrace{(-i \hbar)^{2}}_{\text {collect }}\left[\hat{L}_{x} \hat{L}_{y} \varphi-\hat{L}_{y} \hat{L}_{x}\varphi\right\}=$$

	Apply definition:

	$$(-i \hbar)^{2}\left\{\left(y \frac{\partial}{\partial z}-z \frac{\partial}{\partial y}\right)\left[\left(z \frac{\partial}{\partial x}-x \frac{\partial}{\partial z}\right) \varphi(x, y, z)\right]\right\}-(-i h)^{2}\left\{\left(z \frac{\partial}{\partial x}-x \frac{\partial}{\partial z}\right)\left[\left(y \frac{\partial}{\partial z}-z \frac{\partial}{\partial y}\right) \varphi(x, y, z)\right]\right\}
	$$
	Solving:

	\begin{align*}
		&(-i \hbar)^{2}\left\{\left(y \frac{\partial}{\partial z}-z \frac{\partial}{\partial y}\right)\left[z \frac{\partial}{\partial x} \varphi-x \frac{\partial}{\partial z} \varphi\right]\right\}-(-i h)^{2}\left\{\left(z \frac{\partial}{\partial x}-x \frac{\partial}{\partial z}\right)\left[y \frac{\partial}{\partial z} \varphi-z \frac{\partial}{\partial y} \varphi\right]\right\}=\\
		&=(-i \hbar)^{2}\left\{{y \frac{\partial}{\partial x} \varphi+y z \frac{\partial}{\partial z} \frac{\partial}{\partial x} \varphi-y x \frac{\partial^{2}}{\partial z^{2}} \varphi-z^{2} \frac{\partial}{\partial y} \frac{\partial}{\partial x} \varphi+z x \frac{\partial}{\partial y} \frac{\partial}{\partial z} \varphi}\right\} \\
		&\qquad\qquad-(-i\hbar)^{2}\left\{{z y \frac{\partial}{\partial x} \frac{\partial}{\partial z} \varphi-z^{2} \frac{\partial}{\partial x} \frac{\partial}{\partial y} \varphi-x y \frac{\partial^{2}}{\partial z} \varphi+x \frac{\partial}{\partial y} \varphi+x z \frac{\partial}{\partial z} \frac{\partial}{\partial y} \varphi}\right\}=\\
		&=(-i \hbar)(-i \hbar)\biggl[y \frac{\partial}{\partial x} \varphi-x\frac{\partial}{\partial y}\varphi\biggr] = i\hbar \hat{L}_{z}\varphi \qquad \text {By deleting terms}\\
\end{align*}

	As a result, $\left[\hat{L}_{x}, \hat{L}_{y}\right]=i\hbar \hat{L}_{z}$.\\

	\noindent
	The other relationships are the following:

	\begin{multicols}{2}
		\begin{itemize}
			\item $\left[\hat{L}_{z}, \hat{L}_{x}\right]=-i \hbar \hat{L}_{y}$
			\item $\left[\hat{L}_{y}, \hat{L}_{z}\right]=-i \hbar \hat{L}_{x}$
		\end{itemize}
	\end{multicols}

	Keep in mind that $[\hat{L^2}, \hat{L}_{x}]=0,[\hat{L^2}, \hat{L}_{y}]=0,[\hat{L^2}, \hat{L}_{z}]=0$.\\
	\noindent
	Summarizing,  $\hat{L}_{x},\hat{L}_{y},\hat{L}_{z}$ are not mutually commuting, i.e., we can only know one component of the orbital momentum and the squared modulus (total length of the orbital momentum) at the same time.

\section{Harmonic magnetic trap problem}
In experiments on dilute cold gases, atoms are confined to a certain region of space by a magnetic trap formed by an appropriately configured magnetic field.
Close to the center of the trap, the effect of the magnetic field can be described as a harmonic potential, which is defined by the Hamiltonian.\\
\\
Imagine that we perform an experimental measurement at time $t_0$, giving as outcome $x_{0}=0$ \\
By exploiting this result, we can obtain an approximation for the wave function:
$$\psi(x,t=0)= \mathcal{N} e^{-\frac{m x^{2} \omega}{2 \hbar}}$$
$$
\left(\frac{m \omega}{ \pi \hbar}\right)^{1 / 4}=\mathcal{N}
$$
\noindent
At time $t_0$ the trap is removed and the energy is measured.
We then need to compute the expectation value for $E$:

\begin{align*}
	\expval{E} &=\frac{(\psi, \hat{H} \psi)}{(\psi, \psi)}=\left(\frac{m \omega}{ \pi \hbar}\right)^{\frac{1}{2}} \int_{-\infty}^{+\infty}d x\, \psi^{*}(x) \hat{H} \psi(x)=\\
					 	 &=\left(\frac{m \omega}{ \pi \hbar}\right)^{\frac{1}{2}} \int_{-\infty}^{+\infty} d x\,e^{-\frac{m \omega x^{2}}{2 \hbar}}\left(-\frac{\hbar^{2}}{2 m} \frac{\partial^{2}}{\partial x^{2}}+U(x)\right) e^{-\frac{m \omega x^{2}}{ 2\hbar}}=\\
						 &=\expval{T}+\expval{U}\\
	\expval{T} &=-\frac{\hbar^{2}}{2 m}\left(\frac{m \omega}{\pi \hbar}\right)^{\frac{1}{2}} \int_{-\infty}^{+\infty} d x\, e^{-\frac{m x^{2} \omega}{2 \hbar}} \frac{\partial^{2}}{\partial x^{2}} e^{-\frac{m x^{2} \omega}{2 \hbar}}=\frac{1}{4} \hbar \omega\\
	\expval{U} &=\left(\frac{m \omega}{\pi \hbar}\right)^{\frac{1}{2}} \int_{-\infty}^{+\infty} d x\, e^{-\frac{m x^{2} \omega}{2 \hbar}} U(x) e^{-\frac{m x^{2} \omega}{2 \hbar}}=0
\end{align*}

\noindent
While the trap is present, the function never gets to the infinity boundary, as it gets suppressed beforehand (the trap is narrower than the box).
Therefore, the whole integral equals zero.

$$\expval{U}=0 \rightarrow \expval{E}=\expval{T}=\frac{1}{4} \hbar \omega $$

\noindent
Before $t_0$, the Hamiltonian is the trap.
Let's check if the Hamiltonian at $t_0$ is the box.
Usually, for a free particle in a one dimension box,  $E_n=\left(\frac{\pi^{2}n^2}{L^2}\right)\frac{\hbar^{2}}{2 m}$.\\
\\

	\subsection{Trap larger than the particle}
	By assuming that the trap is much larger than the particle, the particle behaves as a free particle.
	As a consequence, it has many degrees of freedom and we can measure only its kinetic energy, as the probability of finding particles near the edges of the box is way lower than the average value (only kinetic component).\\
	\\
	\noindent
	If the outcome of $\hat{O}$ is $O_n$, given a state $\ket{\psi}$, the probability of finding $O_n$ is:
	\\
	State (initial $\psi): \psi(x)=\left(\frac{m \omega}{ \pi \hbar}\right)^{1 / 4} e^{-\frac{m \omega x^{2}}{2 \hbar}}$, where $P_{n}=|\bra{n} \ket{\psi}|^{2}=\left|\left(f_{0}^{n}(x), \psi(x)\right)\right|^{2}$ overlap integral, with $f_{0}^{n}(x)=\bra{x}\ket{n}$

	\begin{align*}
		P\left(E_{1}\right)&=\left|\int_{-\infty}^{+\infty} d x\, \varphi_{1}^{*}(x) \psi(x)\right|^{2}\qquad \text{ with }\varphi_{1}(x)=\sqrt{\frac{2}{L}} \cos \frac{\pi}{L}x \in \mathbb{R}\\
		&=\left(\int_{-\infty}^{+\infty} d x\,\sqrt{\frac{2}{L}} \cos \frac{\pi}{L} \cdot \left(\frac{m \omega}{ \pi \hbar}\right)^{\frac{1}{2}} e^{-\frac{m \omega x^{2}}{2 \hbar}}\right)^{2}=\\
															&=\frac{m \omega}{ \pi \hbar}\frac{2}{L} \left[\int_{-\infty}^{+\infty} d x \cos \frac{\pi}{L} x e^{-\frac{m \omega x^{2}}{2 \hbar}}\right]^{2}
	\end{align*}

	\noindent

	\subsection{Trap smaller than the box}
	Let's now assume that the trap is much narrower than the box.
	\[\text{Kronecker delta: }  \delta_{i j}=\begin{cases}1,\, i=j \\ 0,\, i \neq j \end{cases} \rightarrow\, \delta(x, x^{\prime}) \sim\begin{cases}1,& \mid x-x^{\prime} \mid<a \\ 0, & x \neq x^{\prime}\end{cases}
	\]
	If $a\ll1$, $1/a\sim 1$\\
	I can also use a Gaussian delta function  $\delta\left(x-x^{\prime}\right)=\frac{1}{\sqrt{2 \pi} \sigma} e^{-\frac{\left(x-x^{\prime}\right)^{2}}{2 \sigma^{2}}}$, which becomes narrower as $\sigma$ gets smaller.\\
	\\
	In our case $\sigma^2= \frac{\hbar}{2m\omega}$, so we can multiply and divide for the Gaussian normalization factor:

	\begin{align*}
		P\left(E_{1}\right)&=\left(\frac{m \omega}{ \pi \hbar}\right)^{\frac{1}{2}} \frac{2}{L}\left(\int_{-\infty}^{+\infty} d x\, \frac{e^{-\frac{x^{2}}{2\sigma^2}}}{\sqrt{2 \pi} \sqrt{\frac{\hbar}{2 m \omega}}} \cos \frac{\pi}{L} x\right)^2 \frac{ \pi \hbar}{m \omega}\\
											 &=\left(\frac{ \pi \hbar}{m\omega}\right) \left(\frac{m \omega}{ \pi \hbar}\right)^{\frac{1}{2}} \frac{2}{L}\int_{-\infty}^{+\infty} d x\, \delta(x-0) \cos \frac{\pi}{L} x=\cos \frac{\pi}{L} \cdot0=1\qquad\text{Applying the Delta function}
	\end{align*}

	\noindent
	Check that $P(E_1)$ is dimensionless:

	$$P\left(E_{1}\right)=\frac{2}{L}\left(\frac{ \pi \hbar}{m \omega}\right)^{\frac{1}{2}} \rightarrow \sqrt{\frac{[J][S]}{[k g][S]^{-1}}}=\sqrt{[\frac{[kg][m]^2[S]^-2[S]}{[kg][S]^-1}} \cdot \frac{1}{[m]}$$

	$P\left(E_{1}\right)$ is dimensionless.

	\subsection{Conclusion}
	The probability has actually a value larger than one, energy levels tend to overlap heavily when considering ground state at small oscillations.
	Anyway, the average value of energy is $\frac{1}{4}\hbar\omega$.
